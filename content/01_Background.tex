\chapter{Background}


- Datendiversität gibts auf unterschiedlichen Ebenen (das muss irgendwie mit drin sein für den Rest der Arbeit)
- Irgendwo auch schonmal irgendwie image naturalness droppen



- Versatile Diffusion \cite{xuVersatileDiffusionText2024} erklären

- Wir wollen einen Bildgenerator der universal ist, also Bilder mit ganz unterschiedlichen Charakteristika rekonstruieren kann. Zum Beispiel auch unabhängig von der Naturalness. Eine Möglichkeit naturalness zu messen ist in Multimedia Features for Click Prediction of New Ads in

- Modernere Bewertung zu was natural ist und was nicht in \cite{chenExploringNaturalnessAIGenerated2023}, folgt zwei verschiedenen Wegen wie der visual pathway

- Bei der Diversity kann man auch darauf eingehen, dass der visual pathway zwei verschiedene Wege folgt, in 20, 31, 66 aus \cite{chenExploringNaturalnessAIGenerated2023}

    - Genauso ist es sehr wichtig, dass eingeführt wird in das Konzept, dass es low-level und high-level Diversity geben kann, sodass die folgenden Kapitel (insbesondere dropout) darauf aufbauen können
    - Das macht ja zum Beispiel auch der ICNN Algorithmus, der die natürliche Hierarchisierung von VGG19 für diese Einteilung nutzt
    - Horikawa und so machen gucken auch schon, wie sich die lowe level Features durch unterschiedliche Gehirnstrukturen vorhersagen lassen