% From mitthesis package
% Version: 1.01, 2023/06/19
% Documentation: https://ctan.org/pkg/mitthesis
%
% The abstract environment creates all the required headers and footnote. 
% You only need to add the text of the abstract itself.
%
% Approximately 500 words or less; try not to use formulas or special characters
% If you don't want an initial indentation, do \noindent at the start of the abstract

\noindent{}Reconstructing images directly from human brain activity was once considered science fiction; today, brain decoding using functional magnetic resonance imaging (fMRI) is turning this concept into reality. However, \rThree{some} current methods \rThree{may face challenges} accurately reconstruct concepts that are not present in the training data. Recent findings in the field have shown that increasing the dimensions and thus the diversity of the training data may lead to better generalisation to unknown concepts. In this paper, three experiments are conducted to investigate the influence of training data diversity on the generalisation of image reconstruction algorithms and the ability to reconstruct images that are dissimilar to the training data. In the first experiment, diversity-based subsampling of the training data was performed to outperform a random subsampling baseline. In the second experiment, image captions were generated to emphasise the low-level (shape, colour) features of the training images, by using this approach the reconstruction of out-of-distribution (OOD) data could be improved. In the third experiment, perturbations were used to either increase or decrease the semantic value of the training data, the results of this experiment were inconclusive. In conclusion, the three experiments showed that the diversity of the dataset can play an important role in helping image reconstruction to generalise to unfamiliar concepts. Further research is needed to find the essential image dimensions needed in the data to create a universal image reconstruction method. It is also important to investigate how these dimensions correspond to the elicited brain activity.

% Yet, the accuracy of these reconstructions critically depends on the diversity of visual stimuli presented during training. This thesis explores how varying the richness and variety of training images influences the brain decoding process and ultimately determines the quality of reconstructed visual experiences.